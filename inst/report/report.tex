\documentclass[]{article}

\usepackage{lmodern}
\usepackage{amssymb,amsmath}
\usepackage{ifxetex,ifluatex}
\ifnum 0\ifxetex 1\fi\ifluatex 1\fi=0 % if pdftex
  \usepackage[T1]{fontenc}
  \usepackage[utf8]{inputenc}
  \usepackage{textcomp} % provide euro and other symbols
\else % if luatex or xetex
  \usepackage{unicode-math}
  \defaultfontfeatures{Scale=MatchLowercase}
  \defaultfontfeatures[\rmfamily]{Ligatures=TeX,Scale=1}
\fi
% Use upquote if available, for straight quotes in verbatim environments
\IfFileExists{upquote.sty}{\usepackage{upquote}}{}
\IfFileExists{microtype.sty}{% use microtype if available
  \usepackage[]{microtype}
  \UseMicrotypeSet[protrusion]{basicmath} % disable protrusion for tt fonts
}{}
\makeatletter
\@ifundefined{KOMAClassName}{% if non-KOMA class
  \IfFileExists{parskip.sty}{%
    \usepackage{parskip}
  }{% else
    \setlength{\parindent}{0pt}
    \setlength{\parskip}{6pt plus 2pt minus 1pt}}
}{% if KOMA class
  \KOMAoptions{parskip=half}}
\makeatother
\usepackage{xcolor}
\IfFileExists{xurl.sty}{\usepackage{xurl}}{} % add URL line breaks if available
\IfFileExists{bookmark.sty}{\usepackage{bookmark}}{\usepackage{hyperref}}
\hypersetup{
  pdftitle={Active Surveillance Progression Prediction},
  pdfauthor={Stefan Eng},
  hidelinks,
  pdfcreator={LaTeX via pandoc}}
\urlstyle{same} % disable monospaced font for URLs
\usepackage{longtable,booktabs}
% Correct order of tables after \paragraph or \subparagraph
\usepackage{etoolbox}
\makeatletter
\patchcmd\longtable{\par}{\if@noskipsec\mbox{}\fi\par}{}{}
\makeatother
% Allow footnotes in longtable head/foot
\IfFileExists{footnotehyper.sty}{\usepackage{footnotehyper}}{\usepackage{footnote}}
\makesavenoteenv{longtable}
\setlength{\emergencystretch}{3em} % prevent overfull lines

\setcounter{secnumdepth}{5}
\usepackage{booktabs}
\usepackage{longtable}
\usepackage{array}
\usepackage{multirow}
\usepackage{wrapfig}
\usepackage{float}
\usepackage{colortbl}
\usepackage{pdflscape}
\usepackage{tabu}
\usepackage{threeparttable}
\usepackage{threeparttablex}
\usepackage[normalem]{ulem}
\usepackage{makecell}
\ifluatex
  \usepackage{selnolig}  % disable illegal ligatures
\fi

% Change tightlist styling: https://tex.stackexchange.com/a/257464/41827
%\providecommand{\tightlist}{%
%  \setlength{\itemsep}{0pt}\setlength{\parskip}{0pt}}

% From template.tex
% include packages needed for your report
\usepackage{graphicx} % to include figures
\usepackage{caption,subcaption} % to include captions and sub captions to figures
\usepackage{titling} % to format title page
\renewcommand\maketitlehooka{\null\mbox{}\vfill} % to center title page
\renewcommand\maketitlehookd{\vfill\null}
\usepackage[margin=1.1in]{geometry} % to adjust the page margin
\usepackage{amsmath} % to display equations
\usepackage{csquotes} % to quote
\usepackage{float} % specify figure and table location
\usepackage{fancyhdr} % to add header
\usepackage{lastpage} % to put the current page number in context of page numbers in the whole document
\usepackage[nottoc,numbib]{tocbibind} % to give section number for the reference
\usepackage{enumitem} % to adjust spacing between bullet point and dashed line
\usepackage{hyperref}

%% Force topsep = 0 for all itemize environments
\setlist[itemize]{topsep=0pt, itemsep=0pt}

\def\tightlist{}

\pagestyle{fancy}
\fancyhf{}
\cfoot{Page \thepage \hspace{0.9pt} of \pageref{LastPage}}

\rhead{Active Surveillance Progression Prediction}
\lhead{\includegraphics[scale=0.02]{uclalogo.png}}


  \title{\Huge Active Surveillance Progression Prediction}
  \author{\\[1cm]
      Stefan Eng\\[1.5cm]
      Boutros Lab\\
      Jonsson Comprehensive Cancer Center\\
      }

% begin report
\begin{document}

% title page
\begin{titlingpage}
\maketitle
\begin{figure}
    \centering
    \includegraphics[scale=0.03]{uclalogo.png}
    \label{fig:ucla_logo}
\end{figure}
\end{titlingpage}

% report outline
\hypertarget{background-to-remove}{%
\section{Background (To Remove)}\label{background-to-remove}}

In general a patient with GS 3+3 (now increasingly called ISUP Grade Group 1 or ISUP GG1 or just GG1) will go on AS while a 3+4 (GG2) may but probably will not and a 4+3 (gg3) definitely will not. Thus you can also look at the patients who did get surgery and see if a GG1 patient on biopsy (a tiny fraction of the tumor) was actually GG2 or GG3 (or worse) after surgery when the entire tumor is available for checking. Thus looking at the surgical pathology can remove the spatial variability component (called undersampling) of the biopsy procedure out.

Clinically, AS means \emph{not} treating a patient. This is of course statistically superior for a bulk population, but still leads to problems when there is a FN (somebody has aggressive disease, but that isn't recognized and thus they are inappropriately on AS). Many patients will therefore voluntarily elect to exit AS prematurely, seeking a treatment that they may not derive benefit from. Thus the key clinical problem is unrecognized aggressive disease, which both hinders those patients directly and more broadly reduces confidence in AS. The goal of our work is thus to give better identification of which patients should exit AS, so that those who should not have more confidence in their decision. Surgery/Radiotherapy (Sx/Rt) is extremely expensive (\$20-50k) and has long-term morbidities (quality-of-life impacts). We sometimes talk about this in terms of \enquote{health preference values}, which reflects how much a change in life-quality relates to a full-year of health. Very roughly, every year on AS relative to definitive local therapy (Sx or Rt) saves the patient \textasciitilde{}0.2 years of fully-healthy life, so prolonging AS as long as possible has big advantages. For example, my dad has been on AS for almost 7 years now, and it's very possible he will be on it for the rest of his life. But, his initial immediate instinct was \enquote{cut this thing out of me}, and having a son who could walk him through the decision-making made a big difference there. So in the long-run, the question is really \enquote{can these biomarkers improve confidence men have to remain on AS} as much as anything else -- \textasciitilde{}15\% each year \emph{voluntarily} leave AS for treatment.

\hypertarget{introduction}{%
\section{Introduction}\label{introduction}}

\hypertarget{study-design-and-objectives}{%
\subsection{Study design and objectives}\label{study-design-and-objectives}}

There is a clinical need to predict \emph{before} surgery if a man has aggressive disease so that we can decide if they need surgery at all. The quality of life benefits are thus huge (avoiding therapy entirely!). The problem is that because it's pre-surgery, we do not have the full cancer to study. Instead we use biopsies (spatially-restricted samples of the cancer), radiology (imaging like MRI) and minimally-invasive biomarkers (e.g.~blood or urine tests). It's unclear which of those different strategies is best, and how those strategies should be sequenced or ordered. A collaborator at UTHSCSA (University of Texas Health Sciences Center San Antonio, I think) Dr.~Michael Liss is a surgeon who is thinking hard about these problems. He's put together a really nice cohort of \textasciitilde{}100 patients where basically every possible biomarker has been generated and we want to figure out \enquote{what is the optimal biomarker we can make using all tests}. That will put an upper-bound to accuracy which we can go investigate in a prospective clinical trial. We can then go backwards to start figuring out if there are ways to simplify/cheapen that test.

\hypertarget{data-description}{%
\subsection{Data description}\label{data-description}}

In this cohort there are \(N = 123\) patients.
There are three variables that are of interest to predict.
The first, \texttt{BiopsyUpgraded}, is whether the patient's research study biopsy increased cancer grade from the most recent biopsy results.
The second, \texttt{ProgressedToTreatment}, is whether the patient progressed to treatment.
Finally we have \texttt{Prostatectomy} which indicates whether the patient had a prostatectomy.

The Prostate Health Index (PHI) and PHI density have been used in detection of prostate cancer (1).

The biomarkers used are shown in \ref{tab:biomarker-categories}.

\begin{table}

\caption{\label{tab:biomarker-categories}Biomarkers categories}
\centering
\begin{tabular}[t]{l|l}
\hline
Category & Biomarker\\
\hline
 & PCA3\\
\cline{2-2}
\multirow{-2}{*}{\raggedright\arraybackslash Urine} & T2ERG\\
\cline{1-2}
 & MiPSCancerRisk\\
\cline{2-2}
\multirow{-2}{*}{\raggedright\arraybackslash Blood/Urine} & MiPSHighGradeCancerRisk\\
\cline{1-2}
 & PSAHyb\\
\cline{2-2}
 & freePSA\\
\cline{2-2}
 & p2PSA\\
\cline{2-2}
 & PercentFreePSA\\
\cline{2-2}
\multirow{-5}{*}{\raggedright\arraybackslash Blood} & PHI\\
\cline{1-2}
 & GeneticAncestry\\
\cline{2-2}
 & GeneticRiskScore\\
\cline{2-2}
 & GeneticRiskCategory\\
\cline{2-2}
 & GlobalScreeningArray\\
\cline{2-2}
\multirow{-5}{*}{\raggedright\arraybackslash Genetics} & GSA Gene\\
\cline{1-2}
 & RSInormalSignal\\
\cline{2-2}
 & RSIlesionSignal\\
\cline{2-2}
 & ADCnormalSignal\\
\cline{2-2}
\multirow{-4}{*}{\raggedright\arraybackslash Imaging} & ADClesionSignal\\
\hline
\end{tabular}
\end{table}

Along with these biomarkers, we computed the PSA Density as \texttt{PSAHyb\ /\ ProstateVolume} and the PHI Density as \texttt{PHI\ /\ ProstateVolume}.

\begin{table}

\caption{\label{tab:prediction-vars}Comparison of prediction variables of interest}
\centering
\begin{tabular}[t]{ccccc}
\toprule
\multicolumn{2}{c}{ } & \multicolumn{2}{c}{Prostatectomy} & \multicolumn{1}{c}{ } \\
\cmidrule(l{3pt}r{3pt}){3-4}
Biopsy Upgraded & Progressed To Treatment & no (N=104) & yes (N=19) & Overall (N=123)\\
\midrule
 & no & 74 & 0 & 74\\
\cmidrule{2-5}
\multirow{-2}{*}{\centering\arraybackslash no} & yes & 5 & 7 & 17\\
\cmidrule{1-5}
 & no & 5 & 0 & 12\\
\cmidrule{2-5}
\multirow{-2}{*}{\centering\arraybackslash yes} & yes & 5 & 12 & 14\\
\cmidrule{1-5}
NA & no & 14 & 0 & 5\\
\bottomrule
\end{tabular}
\end{table}

The severity of the prostate cancers were measured on the Gleason scale/ISUP grade group (2).

\begin{longtable}[]{@{}ll@{}}
\toprule
Gleason Scores & Grade Group\tabularnewline
\midrule
\endhead
No Cancer & Grade Group 0\tabularnewline
Gleason Score \(\leq\) 6 & Grade Group 1\tabularnewline
3 + 4 = 7 & Grade Group 2\tabularnewline
4 + 3 = 7 & Grade Group 3\tabularnewline
Gleason Score 8 & Grade Group 4\tabularnewline
Gleason Score 9-10 & Grade Group 5\tabularnewline
\bottomrule
\end{longtable}

The variable \texttt{PreviousISUP} is the most recent prostate cancer ISUP grade group (GG) rating prior to research biopsy.
The ISUP grade group is computed again in the variable \texttt{StudyHighestISUP} after the research MRI.
We compare the progression of the cancers by the Gleason grades.

\begin{table}

\caption{\label{tab:unnamed-chunk-1}Comparison of pre-research biopsy ISUP grade group to post research MRI ISUP GG}
\centering
\begin{tabular}[t]{r|r|r}
\hline
Decreased & Increased & No Change\\
\hline
36 & 22 & 50\\
\hline
\end{tabular}
\end{table}

\begin{table}[H]
\centering
\begin{tabular}{lccccccc}
\toprule
\multicolumn{1}{c}{ } & \multicolumn{7}{c}{Study Highest ISUP Grade Group } \\
\cmidrule(l{3pt}r{3pt}){2-8}
  & No Cancer & 1 & 2 & 3 & 4 & 5 & NA\\
\midrule
No Cancer & 0 & 0 & 0 & 0 & 0 & 0 & 0\\
1 & 31 & 49 & 13 & 2 & 2 & 1 & 0\\
2 & 2 & 3 & 1 & 2 & 1 & 0 & 0\\
3 & 0 & 0 & 0 & 0 & 1 & 0 & 0\\
4 & 0 & 0 & 0 & 0 & 0 & 0 & 0\\
\addlinespace
5 & 0 & 0 & 0 & 0 & 0 & 0 & 0\\
NA & 0 & 0 & 0 & 0 & 0 & 0 & 15\\
\bottomrule
\end{tabular}
\end{table}

\hypertarget{results}{%
\section{Results}\label{results}}

\hypertarget{descriptive-statistics}{%
\subsection{Descriptive Statistics}\label{descriptive-statistics}}

\hypertarget{demographics}{%
\subsubsection{Demographics}\label{demographics}}

\begin{figure}

{\centering \includegraphics[width=0.9\linewidth]{report_files/figure-latex/unnamed-chunk-2-1} 

}

\caption{Demographics by race and whether biopsy was upgraded.}\label{fig:unnamed-chunk-2}
\end{figure}

\begin{figure}

{\centering \includegraphics[width=0.9\linewidth]{report_files/figure-latex/unnamed-chunk-3-1} 

}

\caption{Demographics by race and whether patient progressed to treatment.}\label{fig:unnamed-chunk-3}
\end{figure}

\begin{tabular}{lccc}
\toprule
\multicolumn{1}{c}{ } & \multicolumn{2}{c}{Progressed To Treatment} & \multicolumn{1}{c}{ } \\
\cmidrule(l{3pt}r{3pt}){2-3}
 & no(N=93) & yes(N=30) & Overall(N=123)\\
\midrule
\addlinespace[0.3em]
\multicolumn{4}{l}{\textbf{[-2]proPSA}}\\
\hspace{1em}Mean (SD) & 14.3 (11.4) & 16.1 (9.87) & 14.7 (11.0)\\
\hspace{1em}Median [Min, Max] & 11.9 [1.73, 75.0] & 13.4 [4.99, 51.0] & 12.8 [1.73, 75.0]\\
\hspace{1em}Missing & 11 (11.8\%) & 4 (13.3\%) & 15 \vphantom{3}(12.2\%)\\
\addlinespace[0.3em]
\multicolumn{4}{l}{\textbf{free PSA}}\\
\hspace{1em}Mean (SD) & 0.937 (0.637) & 0.894 (0.429) & 0.927 (0.592)\\
\hspace{1em}Median [Min, Max] & 0.790 [0.170, 3.87] & 0.845 [0.390, 2.18] & 0.790 [0.170, 3.87]\\
\hspace{1em}Missing & 11 (11.8\%) & 4 (13.3\%) & 15 \vphantom{2}(12.2\%)\\
\addlinespace[0.3em]
\multicolumn{4}{l}{\textbf{PSAHyb}}\\
\hspace{1em}Mean (SD) & 6.81 (4.79) & 7.42 (2.79) & 6.96 (4.39)\\
\hspace{1em}Median [Min, Max] & 5.95 [1.20, 30.4] & 7.40 [2.50, 14.7] & 6.25 [1.20, 30.4]\\
\hspace{1em}Missing & 11 (11.8\%) & 4 (13.3\%) & 15 \vphantom{1}(12.2\%)\\
\addlinespace[0.3em]
\multicolumn{4}{l}{\textbf{PSADensity}}\\
\hspace{1em}Mean (SD) & 0.167 (0.143) & 0.204 (0.0916) & 0.176 (0.132)\\
\hspace{1em}Median [Min, Max] & 0.141 [0.0139, 0.759] & 0.213 [0.0439, 0.374] & 0.145 [0.0139, 0.759]\\
\hspace{1em}Missing & 22 (23.7\%) & 6 (20.0\%) & 28 \vphantom{1}(22.8\%)\\
\addlinespace[0.3em]
\multicolumn{4}{l}{\textbf{SOCPSA}}\\
\hspace{1em}Mean (SD) & 6.82 (7.25) & 7.11 (2.74) & 6.89 (6.44)\\
\hspace{1em}Median [Min, Max] & 5.20 [1.18, 60.0] & 6.34 [3.53, 13.7] & 5.56 [1.18, 60.0]\\
\addlinespace[0.3em]
\multicolumn{4}{l}{\textbf{PHI}}\\
\hspace{1em}Mean (SD) & 39.0 (20.5) & 49.8 (16.4) & 41.6 (20.1)\\
\hspace{1em}Median [Min, Max] & 34.8 [5.40, 101] & 47.1 [24.1, 89.8] & 36.6 [5.40, 101]\\
\hspace{1em}Missing & 11 (11.8\%) & 4 (13.3\%) & 15 (12.2\%)\\
\addlinespace[0.3em]
\multicolumn{4}{l}{\textbf{PHIDensity}}\\
\hspace{1em}Mean (SD) & 0.980 (0.770) & 1.39 (0.640) & 1.08 (0.758)\\
\hspace{1em}Median [Min, Max] & 0.784 [0.0750, 3.89] & 1.39 [0.467, 2.87] & 0.893 [0.0750, 3.89]\\
\hspace{1em}Missing & 22 (23.7\%) & 6 (20.0\%) & 28 (22.8\%)\\
\addlinespace[0.3em]
\multicolumn{4}{l}{\textbf{Prostate Volume (cm\textasciicircum{}3)}}\\
\hspace{1em}Mean (SD) & 50.1 (27.6) & 41.5 (18.4) & 47.9 (25.7)\\
\hspace{1em}Median [Min, Max] & 43.1 [14.8, 191] & 37.9 [19.1, 94.6] & 42.9 [14.8, 191]\\
\hspace{1em}Missing & 15 (16.1\%) & 2 (6.7\%) & 17 (13.8\%)\\
\addlinespace[0.3em]
\multicolumn{4}{l}{\textbf{TNFaAverage}}\\
\hspace{1em}Mean (SD) & 2.75 (0.626) & 2.82 (0.798) & 2.76 (0.668)\\
\hspace{1em}Median [Min, Max] & 2.89 [1.07, 3.66] & 2.92 [1.51, 4.58] & 2.89 [1.07, 4.58]\\
\hspace{1em}Missing & 10 (10.8\%) & 4 (13.3\%) & 14 (11.4\%)\\
\bottomrule
\end{tabular}

\begin{tabular}{lccc}
\toprule
\multicolumn{1}{c}{ } & \multicolumn{2}{c}{Biopsy Upgraded} & \multicolumn{1}{c}{ } \\
\cmidrule(l{3pt}r{3pt}){2-3}
 & no(N=86) & yes(N=22) & Overall(N=123)\\
\midrule
\addlinespace[0.3em]
\multicolumn{4}{l}{\textbf{[-2]proPSA}}\\
\hspace{1em}Mean (SD) & 14.3 (11.5) & 15.4 (8.46) & 14.7 (11.0)\\
\hspace{1em}Median [Min, Max] & 12.2 [1.73, 75.0] & 14.5 [3.07, 29.1] & 12.8 [1.73, 75.0]\\
\hspace{1em}Missing & 8 (9.3\%) & 3 (13.6\%) & 15 \vphantom{3}(12.2\%)\\
\addlinespace[0.3em]
\multicolumn{4}{l}{\textbf{free PSA}}\\
\hspace{1em}Mean (SD) & 0.936 (0.640) & 0.834 (0.388) & 0.927 (0.592)\\
\hspace{1em}Median [Min, Max] & 0.790 [0.170, 3.87] & 0.770 [0.390, 1.64] & 0.790 [0.170, 3.87]\\
\hspace{1em}Missing & 8 (9.3\%) & 3 (13.6\%) & 15 \vphantom{2}(12.2\%)\\
\addlinespace[0.3em]
\multicolumn{4}{l}{\textbf{PSAHyb}}\\
\hspace{1em}Mean (SD) & 6.85 (4.86) & 7.22 (2.74) & 6.96 (4.39)\\
\hspace{1em}Median [Min, Max] & 6.10 [1.20, 30.4] & 7.20 [2.20, 12.5] & 6.25 [1.20, 30.4]\\
\hspace{1em}Missing & 8 (9.3\%) & 3 (13.6\%) & 15 \vphantom{1}(12.2\%)\\
\addlinespace[0.3em]
\multicolumn{4}{l}{\textbf{PSADensity}}\\
\hspace{1em}Mean (SD) & 0.168 (0.135) & 0.211 (0.116) & 0.176 (0.132)\\
\hspace{1em}Median [Min, Max] & 0.144 [0.0139, 0.759] & 0.202 [0.0489, 0.461] & 0.145 [0.0139, 0.759]\\
\hspace{1em}Missing & 9 (10.5\%) & 4 (18.2\%) & 28 \vphantom{1}(22.8\%)\\
\addlinespace[0.3em]
\multicolumn{4}{l}{\textbf{SOCPSA}}\\
\hspace{1em}Mean (SD) & 7.06 (7.44) & 6.88 (3.06) & 6.89 (6.44)\\
\hspace{1em}Median [Min, Max] & 5.52 [1.18, 60.0] & 6.13 [2.19, 13.7] & 5.56 [1.18, 60.0]\\
\addlinespace[0.3em]
\multicolumn{4}{l}{\textbf{PHI}}\\
\hspace{1em}Mean (SD) & 39.6 (19.8) & 47.8 (17.4) & 41.6 (20.1)\\
\hspace{1em}Median [Min, Max] & 35.5 [5.40, 97.0] & 47.3 [11.8, 84.2] & 36.6 [5.40, 101]\\
\hspace{1em}Missing & 8 (9.3\%) & 3 (13.6\%) & 15 (12.2\%)\\
\addlinespace[0.3em]
\multicolumn{4}{l}{\textbf{PHIDensity}}\\
\hspace{1em}Mean (SD) & 1.01 (0.758) & 1.39 (0.695) & 1.08 (0.758)\\
\hspace{1em}Median [Min, Max] & 0.846 [0.0750, 3.89] & 1.41 [0.262, 2.48] & 0.893 [0.0750, 3.89]\\
\hspace{1em}Missing & 9 (10.5\%) & 4 (18.2\%) & 28 (22.8\%)\\
\addlinespace[0.3em]
\multicolumn{4}{l}{\textbf{Prostate Volume (cm\textasciicircum{}3)}}\\
\hspace{1em}Mean (SD) & 49.5 (27.1) & 41.3 (18.2) & 47.9 (25.7)\\
\hspace{1em}Median [Min, Max] & 43.0 [18.0, 191] & 40.0 [14.8, 79.1] & 42.9 [14.8, 191]\\
\hspace{1em}Missing & 1 (1.2\%) & 1 (4.5\%) & 17 (13.8\%)\\
\addlinespace[0.3em]
\multicolumn{4}{l}{\textbf{TNFaAverage}}\\
\hspace{1em}Mean (SD) & 2.76 (0.681) & 2.64 (0.634) & 2.76 (0.668)\\
\hspace{1em}Median [Min, Max] & 2.89 [1.07, 4.58] & 2.95 [1.51, 3.72] & 2.89 [1.07, 4.58]\\
\hspace{1em}Missing & 7 (8.1\%) & 3 (13.6\%) & 14 (11.4\%)\\
\bottomrule
\end{tabular}

\begin{figure}

{\centering \includegraphics[width=1\linewidth]{report_files/figure-latex/unnamed-chunk-6-1} 

}

\caption{Correlation heapmap with test with test methodology labels.}\label{fig:unnamed-chunk-6}
\end{figure}

\hypertarget{time-to-event-data}{%
\subsubsection{Time-to-event data}\label{time-to-event-data}}

All patients have values for the days from diagnosis to the last clinical appointment \texttt{DaysDxToLastClinicalAppt} and last review \texttt{DaysDxToLastReview}.
The patients that had a biopsy have values for variables \texttt{DaysBxToLastClinicalAppt} and \texttt{DaysBxToLastReview}.
If a patient progressed to treatment, then the days from diagnosis are in \texttt{DaysDxToProgression}.
If they had a biopsy and progressed to treatment then \texttt{DaysDxToProgression} will contain the days between the biopsy and progression.

\hypertarget{overall-days-to-progression-from-biopsydiagnosis}{%
\paragraph{Overall Days to Progression from Biopsy/Diagnosis}\label{overall-days-to-progression-from-biopsydiagnosis}}

\begin{figure}

{\centering \includegraphics[width=0.6\linewidth]{report_files/figure-latex/days-bx-progression-1} 

}

\caption{Days to progression from biopsy for entire cohort.}\label{fig:days-bx-progression}
\end{figure}

\begin{center}\includegraphics[width=0.6\linewidth]{report_files/figure-latex/unnamed-chunk-8-1} \end{center}

\hypertarget{overall-time-to-upgrade-from-diagnosis}{%
\paragraph{Overall Time-to-Upgrade from Diagnosis}\label{overall-time-to-upgrade-from-diagnosis}}

\begin{center}\includegraphics[width=0.6\linewidth]{report_files/figure-latex/unnamed-chunk-9-1} \end{center}

\hypertarget{predictive-analysis}{%
\subsection{Predictive Analysis}\label{predictive-analysis}}

We predict the three variables: \texttt{ProgressedToTreatment}, \texttt{BiopsyUpgraded}, \texttt{Prostatectomy}.
For each of these variable we compare the results from three models: a regression tree (\texttt{rpart}), gradient boosting machine (\texttt{gbm}), and eXtreme Gradient Boosting (\texttt{xgboost}).
We only report the results from \texttt{rpart} and \texttt{gbm} as the results are very similar for the two gradient boosting methods and \texttt{gbm} performs slightly better.
The regression tree model is the most interpretable as it produces a single decision tree.
The other models produce better accuracy as well as \(F_1\) scores but require more work to interpret.

All of the models were validated using 10-fold cross-validation repeated 5 times.
That is, we partition the data set into 10 parts and use 9 parts to train the model and the last to evaluate the model.
Several metrics are computed based on the results such as \(F_1\) score, area under the ROC curve (ROC-AUC), area under the precision-recall curve, accuracy, sensitivity, and specificity.
This process is repeated so that each part of the partition is used to validate exactly once.
We repeat this entire process 5 times so that 10 random partitions are generated.
This results in a distribution for the metrics which is an approximation for how to models will perform out of sample.
For each of the models we performed a grid-search to find the optimal parameters.
The area under the precision-recall curve was used as the (threshold invariant) metric for which the parameters were tuned.

The variables in \ref{tab:variable-set} were used as the initial set of variables for \texttt{ProgressedToTreatment} and \texttt{Prostatectomy}. In the case of \texttt{BiopsyUpgraded}, the variable \texttt{BiopsyResult} was removed from the predictor variables.

\begin{table}

\caption{\label{tab:variable-set}Variables used to predict ProgressedToTreatment and Prostatectomy}
\centering
\begin{tabular}[t]{ll}
\toprule
\multicolumn{2}{c}{Variables} \\
\cmidrule(l{3pt}r{3pt}){1-2}
Age & PHI\\
Race & PreviousISUP\\
Hispanic & GeneticRiskScore\\
Weight & TNFaAverage\\
Height & TNFaSTD\\
\addlinespace
BMI & GlobalScreeningArray\\
MRIResult & GSAPositives\\
MRILesions & BRCAMutation\\
BiopsyResult & Mutation\_BRCA1\\
ProstateVolume & Mutation\_BRCA2\\
\addlinespace
PCA3 & Mutation\_ATM\\
T2ERG & RSInormalSignal\\
MiPSCancerRisk & RSIlesionSignal\\
MiPSHighGradeCancerRisk & ADCnormalSignal\\
PSAHyb & ADClesionSignal\\
\addlinespace
freePSA & PSADensity\\
p2PSA & PHIDensity\\
PercentFreePSA & \\
\bottomrule
\end{tabular}
\end{table}

\hypertarget{roc-and-pr-curves}{%
\subsubsection{ROC and PR Curves}\label{roc-and-pr-curves}}

For each of the targets, \texttt{ProgressedToTreatment}, \texttt{BiopsyUpgraded}, and \texttt{Prostatectomy} we construct Receiver operating characteristic (ROC) and Precision-Recall curves.
The sensitivity/recall, specificity, and precision are all computed from the cross-validation folds to give an estimate of out of sample performance.

\begin{figure}

{\centering \includegraphics[width=0.9\linewidth]{report_files/figure-latex/prog-roc-1} 

}

\caption{Cross-Validation ROC and Precision-Recall Curves for Progressed to Treatment. The optimal (Youden's J statistic) is indicated with a point on the ROC curve.}\label{fig:prog-roc}
\end{figure}

\begin{figure}

{\centering \includegraphics[width=0.9\linewidth]{report_files/figure-latex/bx-roc-1} 

}

\caption{Cross-Validation ROC and Precision-Recall Curves for Biopsy Upgraded. The optimal (Youden's J statistic) is indicated with a point on the ROC curve.}\label{fig:bx-roc}
\end{figure}

\begin{figure}

{\centering \includegraphics[width=0.9\linewidth]{report_files/figure-latex/prostate-roc-1} 

}

\caption{Cross-Validation ROC and Precision-Recall Curves for Prostatectomy. The optimal (Youden's J statistic) is indicated with a point on the ROC curve.}\label{fig:prostate-roc}
\end{figure}

\hypertarget{optimal-operating-point}{%
\subsubsection{Optimal Operating Point}\label{optimal-operating-point}}

Each of the models outputs a probability being in class 1, e.g.~the probability of biopsy upgraded, or progression to treatment.
By default the models threshold anything above 0.5 as class 1 and anything below as 0.
We want to find a better threshold value, called the operating point.

The first method we test for finding the optimal operating point is to maximize Youden's J statistic (3) which is defined as
\[
J = \text{sensitivity} + \text{specificity} - 1
\]
The second method we use to find the optimal operating point is to make a false negative FN (when we predict indolent disease when it truly is aggressive) 2 times as costly as a false positive FP.
For each threshold the cost is computed for each of the folds in the 10-fold cross-validation and averaged across all 10 folds and 5 repetitions.
Then the optimal operating point is selected as the threshold with the minimum cost.

We show how to find the optimal point for predicting whether a patient progressed to treatment.

\begin{figure}

{\centering \includegraphics[width=0.6\linewidth]{report_files/figure-latex/unnamed-chunk-12-1} 

}

\caption{Optimal operating point for the Progressed to Treatment prediction models.}\label{fig:unnamed-chunk-12}
\end{figure}

The results are very similar for the \(J\) statistic optimization and the cost-weighted method.
The only model in which the results differ dramatically are in the prostatectomy prediction.
In the cost-weighted method the threshold is optimized at 0 or 1 which results in all positive or all negative predictions.
Using the \(J\) statistic prevents the models from predicting the same class.
We show summary statistics in \ref{tab:threshold-results} for the optimal thresholds for both J statistic and cost-weighted methods.

\begin{table}

\caption{\label{tab:threshold-results}Summary statistics for the different operating points, targets and models.}
\centering
\begin{tabular}[t]{>{}l|l|l|r|r|r|r|r|r}
\hline
target & model & method & threshold & Accuracy & Sensitivity & Specificity & Precision & F1\\
\hline
 &  & cost & 0.11 & 0.64 & 0.70 & 0.61 & 0.55 & 0.55\\
\cline{3-9}
 & \multirow{-2}{*}{\raggedright\arraybackslash rpart} & roc & 0.16 & 0.64 & 0.70 & 0.61 & 0.55 & 0.55\\
\cline{2-9}
 &  & cost & 0.37 & 0.85 & 0.64 & 0.91 & 0.74 & 0.71\\
\cline{3-9}
 & \multirow{-2}{*}{\raggedright\arraybackslash gbm} & roc & 0.37 & 0.85 & 0.64 & 0.91 & 0.74 & 0.71\\
\cline{2-9}
 &  & cost & 0.21 & 0.82 & 0.68 & 0.86 & 0.65 & 0.67\\
\cline{3-9}
\multirow{-6}{*}{\raggedright\arraybackslash \makecell[c]{Biopsy\\Upgraded}} & \multirow{-2}{*}{\raggedright\arraybackslash xgbTree} & roc & 0.09 & 0.79 & 0.77 & 0.80 & 0.58 & 0.67\\
\cline{1-9}
 &  & cost & 0.71 & 0.89 & 0.04 & 0.99 & 0.33 & 0.67\\
\cline{3-9}
 & \multirow{-2}{*}{\raggedright\arraybackslash rpart} & roc & 0.04 & 0.57 & 0.75 & 0.54 & 0.24 & 0.39\\
\cline{2-9}
 &  & cost & 0.01 & 0.11 & 1.00 & 0.00 & 0.11 & 0.19\\
\cline{3-9}
 & \multirow{-2}{*}{\raggedright\arraybackslash gbm} & roc & 0.11 & 0.79 & 0.76 & 0.80 & 0.37 & 0.55\\
\cline{2-9}
 &  & cost & 0.95 & 0.89 & 0.00 & 1.00 & NaN & NaN\\
\cline{3-9}
\multirow{-6}{*}{\raggedright\arraybackslash Prostatectomy} & \multirow{-2}{*}{\raggedright\arraybackslash xgbTree} & roc & 0.02 & 0.60 & 0.82 & 0.58 & 0.19 & 0.35\\
\cline{1-9}
 &  & cost & 0.31 & 0.85 & 0.54 & 0.91 & 0.58 & 0.66\\
\cline{3-9}
 & \multirow{-2}{*}{\raggedright\arraybackslash rpart} & roc & 0.33 & 0.85 & 0.53 & 0.91 & 0.59 & 0.66\\
\cline{2-9}
 &  & cost & 0.21 & 0.86 & 0.66 & 0.89 & 0.59 & 0.66\\
\cline{3-9}
 & \multirow{-2}{*}{\raggedright\arraybackslash gbm} & roc & 0.21 & 0.86 & 0.71 & 0.89 & 0.60 & 0.68\\
\cline{2-9}
 &  & cost & 0.39 & 0.85 & 0.39 & 0.93 & 0.54 & 0.66\\
\cline{3-9}
\multirow{-6}{*}{\raggedright\arraybackslash \makecell[c]{Progressed\\To\\Treatment}} & \multirow{-2}{*}{\raggedright\arraybackslash xgbTree} & roc & 0.10 & 0.82 & 0.58 & 0.86 & 0.51 & 0.63\\
\hline
\end{tabular}
\end{table}

\hypertarget{cut-points-for-time-to-event-analysis}{%
\subsection{Cut-points for time-to-event analysis}\label{cut-points-for-time-to-event-analysis}}

The gradient boosted model (gbm) provided the best predictive power of the models.
We use the predictions from this model to show that days until the patient progressed or had the biopsy upgraded for each of the prediction groups.

\hypertarget{progressed-to-treatment}{%
\subsubsection{Progressed To Treatment}\label{progressed-to-treatment}}

\begin{figure}

{\centering \includegraphics[width=0.6\linewidth]{report_files/figure-latex/unnamed-chunk-13-1} 

}

\caption{Days to progression from biopsy for the prediction groups}\label{fig:unnamed-chunk-13}
\end{figure}

\begin{figure}

{\centering \includegraphics[width=0.6\linewidth]{report_files/figure-latex/unnamed-chunk-14-1} 

}

\caption{Days to progression from diagnosis for the prediction groups}\label{fig:unnamed-chunk-14}
\end{figure}

\hypertarget{days-to-upgrade-from-diagnosis}{%
\subsubsection{Days to Upgrade from Diagnosis}\label{days-to-upgrade-from-diagnosis}}

\begin{center}\includegraphics[width=0.6\linewidth]{report_files/figure-latex/unnamed-chunk-15-1} \end{center}

\hypertarget{variable-importance-for-progressed-to-treatment}{%
\subsection{Variable Importance for Progressed to Treatment}\label{variable-importance-for-progressed-to-treatment}}

The regression tree output is simple to interpret but does not provide the best predictions.
The GBM is not as easy to interpret but has a measure of variable importance of the predictors.
We compare the models' variable importance ranks to see important variables across the models.

\begin{table}

\caption{\label{tab:unnamed-chunk-16}Top 25 variable importance rankings out of 35 variables}
\centering
\begin{tabular}[t]{l|r|r|r|r}
\hline
  & Mean rank & GBM rank & XGB rank & rpart rank\\
\hline
RSIlesionSignal & 1.00 & 1 & 1.0 & 1.0\\
\hline
TNFaSTD & 2.67 & 2 & 2.0 & 4.0\\
\hline
Age & 6.33 & 11 & 5.0 & 3.0\\
\hline
PHIDensity & 7.67 & 3 & 18.0 & 2.0\\
\hline
PercentFreePSA & 9.50 & 5 & 3.0 & 20.5\\
\hline
TNFaAverage & 10.83 & 6 & 6.0 & 20.5\\
\hline
RSInormalSignal & 11.17 & 9 & 4.0 & 20.5\\
\hline
BiopsyResult & 11.67 & 10 & 20.0 & 5.0\\
\hline
ADClesionSignal & 11.83 & 8 & 7.0 & 20.5\\
\hline
ADCnormalSignal & 11.83 & 4 & 11.0 & 20.5\\
\hline
PHI & 16.17 & 7 & 21.0 & 20.5\\
\hline
T2ERG & 17.50 & 24 & 8.0 & 20.5\\
\hline
PSAHyb & 17.83 & 24 & 9.0 & 20.5\\
\hline
GeneticRiskScore & 18.17 & 24 & 10.0 & 20.5\\
\hline
MiPSCancerRisk & 18.83 & 24 & 12.0 & 20.5\\
\hline
MiPSHighGradeCancerRisk & 19.17 & 24 & 13.0 & 20.5\\
\hline
BMI & 19.50 & 24 & 14.0 & 20.5\\
\hline
Height & 19.83 & 24 & 15.0 & 20.5\\
\hline
PCA3 & 20.17 & 24 & 16.0 & 20.5\\
\hline
PSADensity & 20.50 & 24 & 17.0 & 20.5\\
\hline
Hispanic & 21.00 & 12 & 30.5 & 20.5\\
\hline
PreviousISUP & 21.17 & 24 & 19.0 & 20.5\\
\hline
ProstateVolume & 22.17 & 24 & 22.0 & 20.5\\
\hline
freePSA & 22.50 & 24 & 23.0 & 20.5\\
\hline
p2PSA & 22.83 & 24 & 24.0 & 20.5\\
\hline
\end{tabular}
\end{table}

\hypertarget{decision-trees}{%
\subsection{Decision Trees}\label{decision-trees}}

The GBM models perform much better than the rpart models.
We include the rpart decision tree models as they are easy to interpret.

\begin{figure}

{\centering \includegraphics[width=0.75\linewidth]{report_files/figure-latex/decision-tree-bx-1} 

}

\caption{Decision tree for Biopsy Upgraded prediction.}\label{fig:decision-tree-bx}
\end{figure}

\begin{figure}

{\centering \includegraphics[width=0.9\linewidth]{report_files/figure-latex/decision-tree-prog-1} 

}

\caption{Decision tree for Progressed to Treatment prediction.}\label{fig:decision-tree-prog}
\end{figure}

\begin{figure}

{\centering \includegraphics[width=0.75\linewidth]{report_files/figure-latex/decision-tree-prostate-1} 

}

\caption{Decision tree for Prostatectomy prediction.}\label{fig:decision-tree-prostate}
\end{figure}

\hypertarget{next-steps}{%
\section{Next steps}\label{next-steps}}

\begin{itemize}
\tightlist
\item
  Describe the following analysis steps
\item
  Propose solutions to address constraints or opportunities uncovered during the analysis
\end{itemize}

\hypertarget{conclusiondiscussion}{%
\section{Conclusion/Discussion}\label{conclusiondiscussion}}

\begin{itemize}
\tightlist
\item
  Optional
\item
  Describe the patterns, principles, and relationships shown by each major findings
\item
  Describe potential limitations and/or new questions
\end{itemize}

\hypertarget{methods}{%
\section{Methods}\label{methods}}

\begin{itemize}
\tightlist
\item
  The tools/software you are using and cite version of software and packages used

  \begin{itemize}
  \tightlist
  \item
    R language and environment (R Core Team 2019, v3.6.0)
  \item
    BPG (v5.9.8) (4) (to embed citation within text)
  \end{itemize}
\item
  Describe handling dropouts and missing data
\end{itemize}

\begin{verbatim}
## - Session info -------------------------------------------------------------------------------------------------------
##  setting  value                       
##  version  R version 3.6.3 (2020-02-29)
##  os       macOS Mojave 10.14.6        
##  system   x86_64, darwin15.6.0        
##  ui       X11                         
##  language (EN)                        
##  collate  en_US.UTF-8                 
##  ctype    en_US.UTF-8                 
##  tz       America/Los_Angeles         
##  date     2020-10-09                  
## 
## - Packages -----------------------------------------------------------------------------------------------------------
##  package                                * version    date       lib source        
##  assertthat                               0.2.1      2019-03-21 [1] CRAN (R 3.6.0)
##  backports                                1.1.5      2019-10-02 [1] CRAN (R 3.6.0)
##  bookdown                                 0.11       2019-05-28 [1] CRAN (R 3.6.0)
##  BoutrosLab.plotting.general            * 6.0.1      2020-08-28 [1] local         
##  BoutrosLab.plotting.survival             3.0.10     2020-09-15 [1] local         
##  BoutrosLab.prognosticsignature.general   1.3.13     2020-09-15 [1] local         
##  BoutrosLab.statistics.general            2.1.3      2020-08-26 [1] local         
##  BoutrosLab.statistics.survival           0.4.20     2020-09-15 [1] local         
##  BoutrosLab.utilities                     1.9.10     2020-08-26 [1] local         
##  caret                                  * 6.0-86     2020-03-20 [1] CRAN (R 3.6.0)
##  class                                    7.3-15     2019-01-01 [1] CRAN (R 3.6.3)
##  cli                                      2.0.2      2020-02-28 [1] CRAN (R 3.6.0)
##  cluster                                * 2.1.0      2019-06-19 [1] CRAN (R 3.6.3)
##  codetools                                0.2-16     2018-12-24 [1] CRAN (R 3.6.3)
##  colorspace                               1.4-1      2019-03-18 [1] CRAN (R 3.6.0)
##  crayon                                   1.3.4      2017-09-16 [1] CRAN (R 3.6.0)
##  data.table                               1.12.8     2019-12-09 [1] CRAN (R 3.6.0)
##  digest                                   0.6.25     2020-02-23 [1] CRAN (R 3.6.0)
##  doParallel                               1.0.15     2019-08-02 [1] CRAN (R 3.6.0)
##  dplyr                                    1.0.2      2020-08-18 [1] CRAN (R 3.6.2)
##  e1071                                    1.7-1      2019-03-19 [1] CRAN (R 3.6.0)
##  evaluate                                 0.14       2019-05-28 [1] CRAN (R 3.6.0)
##  fansi                                    0.4.1      2020-01-08 [1] CRAN (R 3.6.0)
##  foreach                                  1.4.4      2017-12-12 [1] CRAN (R 3.6.0)
##  Formula                                  1.2-3      2018-05-03 [1] CRAN (R 3.6.0)
##  gbm                                      2.1.8      2020-07-15 [1] CRAN (R 3.6.2)
##  generics                                 0.0.2      2018-11-29 [1] CRAN (R 3.6.0)
##  ggplot2                                * 3.3.1      2020-05-28 [1] CRAN (R 3.6.2)
##  glue                                     1.4.1      2020-05-13 [1] CRAN (R 3.6.2)
##  gower                                    0.2.0      2019-03-07 [1] CRAN (R 3.6.0)
##  gridExtra                              * 2.3        2017-09-09 [1] CRAN (R 3.6.0)
##  gtable                                   0.3.0      2019-03-25 [1] CRAN (R 3.6.0)
##  here                                   * 0.1        2017-05-28 [1] CRAN (R 3.6.0)
##  hexbin                                 * 1.27.3     2019-05-14 [1] CRAN (R 3.6.0)
##  htmltools                                0.4.0      2019-10-04 [1] CRAN (R 3.6.0)
##  httr                                     1.4.1      2019-08-05 [1] CRAN (R 3.6.0)
##  ipred                                    0.9-9      2019-04-28 [1] CRAN (R 3.6.0)
##  iterators                                1.0.10     2018-07-13 [1] CRAN (R 3.6.0)
##  kableExtra                             * 1.2.1      2020-08-27 [1] CRAN (R 3.6.2)
##  knitr                                    1.28       2020-02-06 [1] CRAN (R 3.6.0)
##  lattice                                * 0.20-38    2018-11-04 [1] CRAN (R 3.6.3)
##  latticeExtra                           * 0.6-28     2016-02-09 [1] CRAN (R 3.6.0)
##  lava                                     1.6.5      2019-02-12 [1] CRAN (R 3.6.0)
##  lifecycle                                0.2.0      2020-03-06 [1] CRAN (R 3.6.0)
##  lubridate                                1.7.4      2018-04-11 [1] CRAN (R 3.6.0)
##  magrittr                               * 1.5        2014-11-22 [1] CRAN (R 3.6.0)
##  MASS                                     7.3-51.5   2019-12-20 [1] CRAN (R 3.6.3)
##  Matrix                                   1.2-18     2019-11-27 [1] CRAN (R 3.6.3)
##  ModelMetrics                             1.2.2.2    2020-03-17 [1] CRAN (R 3.6.0)
##  munsell                                  0.5.0      2018-06-12 [1] CRAN (R 3.6.0)
##  nlme                                     3.1-144    2020-02-06 [1] CRAN (R 3.6.3)
##  nnet                                     7.3-12     2016-02-02 [1] CRAN (R 3.6.3)
##  pillar                                   1.4.3      2019-12-20 [1] CRAN (R 3.6.0)
##  pkgconfig                                2.0.3      2019-09-22 [1] CRAN (R 3.6.0)
##  plyr                                     1.8.6      2020-03-03 [1] CRAN (R 3.6.0)
##  pROC                                   * 1.16.2     2020-03-19 [1] CRAN (R 3.6.0)
##  prodlim                                  2018.04.18 2018-04-18 [1] CRAN (R 3.6.0)
##  ProstateCancer.ASBiomarkerSynergy      * 0.0.0.9000 2020-10-09 [1] local         
##  purrr                                    0.3.3      2019-10-18 [1] CRAN (R 3.6.0)
##  R6                                       2.4.1      2019-11-12 [1] CRAN (R 3.6.0)
##  rbenchmark                               1.0.0      2012-08-30 [1] CRAN (R 3.6.0)
##  RColorBrewer                           * 1.1-2      2014-12-07 [1] CRAN (R 3.6.0)
##  Rcpp                                     1.0.3      2019-11-08 [1] CRAN (R 3.6.0)
##  recipes                                  0.1.13     2020-06-23 [1] CRAN (R 3.6.2)
##  reshape2                                 1.4.3      2017-12-11 [1] CRAN (R 3.6.0)
##  rJava                                    0.9-13     2020-07-06 [1] CRAN (R 3.6.2)
##  rlang                                    0.4.7      2020-07-09 [1] CRAN (R 3.6.2)
##  rmarkdown                                2.1        2020-01-20 [1] CRAN (R 3.6.0)
##  rpart                                  * 4.1-15     2019-04-12 [1] CRAN (R 3.6.0)
##  rpart.plot                             * 3.0.8      2019-08-22 [1] CRAN (R 3.6.0)
##  rprojroot                                1.3-2      2018-01-03 [1] CRAN (R 3.6.0)
##  rstudioapi                               0.11       2020-02-07 [1] CRAN (R 3.6.0)
##  rvest                                    0.3.5      2019-11-08 [1] CRAN (R 3.6.0)
##  scales                                   1.1.0      2019-11-18 [1] CRAN (R 3.6.0)
##  selectr                                  0.4-2      2019-11-20 [1] CRAN (R 3.6.0)
##  sessioninfo                              1.1.1      2018-11-05 [1] CRAN (R 3.6.0)
##  stringi                                  1.4.6      2020-02-17 [1] CRAN (R 3.6.0)
##  stringr                                  1.4.0      2019-02-10 [1] CRAN (R 3.6.0)
##  survival                                 3.1-8      2019-12-03 [1] CRAN (R 3.6.3)
##  table1                                   1.2        2020-03-23 [1] CRAN (R 3.6.0)
##  tibble                                   2.1.3      2019-06-06 [1] CRAN (R 3.6.0)
##  tidyselect                               1.1.0      2020-05-11 [1] CRAN (R 3.6.2)
##  timeDate                                 3043.102   2018-02-21 [1] CRAN (R 3.6.0)
##  vctrs                                    0.3.4      2020-08-29 [1] CRAN (R 3.6.2)
##  viridisLite                              0.3.0      2018-02-01 [1] CRAN (R 3.6.0)
##  webshot                                  0.5.2      2019-11-22 [1] CRAN (R 3.6.0)
##  withr                                    2.1.2      2018-03-15 [1] CRAN (R 3.6.0)
##  xfun                                     0.12       2020-01-13 [1] CRAN (R 3.6.0)
##  xgboost                                  1.2.0.1    2020-09-02 [1] CRAN (R 3.6.2)
##  xlsx                                     0.6.3      2020-02-28 [1] CRAN (R 3.6.0)
##  xlsxjars                                 0.6.1      2014-08-22 [1] CRAN (R 3.6.0)
##  xml2                                     1.2.5      2020-03-11 [1] CRAN (R 3.6.0)
##  yaml                                     2.2.1      2020-02-01 [1] CRAN (R 3.6.0)
## 
## [1] /Library/Frameworks/R.framework/Versions/3.6/Resources/library
\end{verbatim}

\newpage

\hypertarget{references}{%
\section{References}\label{references}}

\hypertarget{refs}{}
\leavevmode\hypertarget{ref-Druskin2017}{}%
1. Tosoian, J. J. \emph{et al.} Prostate health index density improves detection of clinically significant prostate cancer. \emph{BJU International} \textbf{120,} 793--798 (2017).

\leavevmode\hypertarget{ref-ISUP2016}{}%
2. Egevad, L., Delahunt, B., Srigley, J. R. \& Samaratunga, H. International society of urological pathology (isup) grading of prostate cancer -- an isup consensus on contemporary grading. \emph{APMIS} \textbf{124,} 433--435 (2016).

\leavevmode\hypertarget{ref-Youden1950}{}%
3. Youden, W. J. Index for rating diagnostic tests. \emph{Cancer} \textbf{3,} 32--35 (1950).

\leavevmode\hypertarget{ref-BPG}{}%
4. Christine P'ng, L. C. C., Jeffrey Green. BPG: Seamless, automated and interactive visualization of scientific data. \emph{BMC Bioinformatics} (2019). doi:\href{https://doi.org/https://doi.org/10.1186/s12859-019-2610-2}{https://doi.org/10.1186/s12859-019-2610-2}

\newpage

\hypertarget{appendix}{%
\section{Appendix}\label{appendix}}

\begin{itemize}
\tightlist
\item
  Optional
\item
  Include supplementary materials that may be helpful in providing a more comprehensive understanding of the research problem.
\item
  Technical description of statistical procedures
\item
  Detailed tables or computer output
\item
  Figures that were not central to the arguments presented in the body of the reportot.
\end{itemize}

\bibliography{bibliography.bib}

\end{document}
